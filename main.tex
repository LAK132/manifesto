\documentclass{article}
\usepackage{graphicx}
\usepackage{listings}

\title{Ripping Up The Garden Walls}
\author{LAK132}

\begin{document}

\maketitle

\tableofcontents

\section[trees]{Trees}

Everything is a tree.

\section[computer-aided-design]{Computer Aided Design}

\cite{can-a-machine-design}

\section[universal-theory-of-computing]{Universal Theory of Computing}

\cite{the-unix-operating-system}
\cite{what-unix-cost-us}

\subsection[computational-computing]{Computational Computing}

Theory

\subsection[personal-computing]{Personal Computing}

Machine

\subsection[augmentative-computing]{Augmentative Computing}

Tool

\section[authority]{Authority}

\cite{the-tragedy-of-systemd}

\section[the-human-problem]{The Human Problem}

\cite{augmentating-human-intellect}
\cite{mother-of-all-demos}
\cite{augmentation-of-douglas-engelbart}
\cite{written-for-eyes}
\cite{the-future-of-programming}
\cite{a-few-words-on-doug-engelbart}
\cite{inventing-on-principle}

\section[object-oriented-programming]{Object Oriented Programming}

The issue with OOP is not the Objects, but the Orientation.

\section[compression-oriented-programming]{Compression Oriented Programming}

\cite{designing-and-evaluating-reusable-components}
\cite{immediate-mode-gui}
\cite{five-second-stall}
\cite{semantic-compression}
\cite{complexity-and-granularity}
\cite{thirty-million-line-problem}
\cite{twitter-and-visual-studio-rant}
\cite{how-fast-should-an-unoptimised-terminal-run}
\cite{simple-code-high-performance}
\cite{where-does-bad-code-come-from}
\cite{the-only-unbreakable-law}

Blender's node programming demonstrates compression oriented programming
through the ability to group a selection of nodes into a new node.

\section[type-oriented-programming]{Type Oriented Programming}

Use types only to make invalid states unrepresentable,
or as a granular\cite{complexity-and-granularity} step to
improving/simplifying an API,
not as encapsulation for encapsulations sake.

Does not contradict \ref{compression-oriented-programming},
instead type oriented programming describes a set of tools one may use while
attempting to compress or debug code.
We aim to use the language's type system to systematically eradicate invalid
state and therefore catch bugs.

\subsection[space-types]{Space Types}

Something that comes up a lot in computer graphics is the `space' that a
particular set of points, vectors and transformations operate in.
Be that Model space, View space, World space, etc.
Transformations between these spaces are generally named as such,
for example, the Model View Projection Matrix (MVP for short) translates points
and such from the Model space to Screen/Clip space.

On a project I worked on a while ago, I discovered an issue where points in
one space were being treated as if they were actually in a different space,
resulting in numerous rendering issues.
I decided to solve this issue through use of the type system:
if a point truly exists in Model space, its type should be tagged as such.
Transformations between different spaces would now also change the type tag.
This caught more bugs than I had originally expected.
Importantly though, the new system was design to allow the programmer to
\emph{explicitly} sidestep type system, to effectively allow for \emph{casting}
between spaces.
This meant that code that was intentionally jumping between spaces without
a transformation could continue doing so,
but now made it explicitly clear that that was happening.
We moved the burden of validating space transformations to the compiler,
whilst still maintaining enough granularity\cite{complexity-and-granularity}
to allow the programmer to do their job without having to jump through a
million hoops.

\section[misc]{Misc}

\cite{100-rabbits}
\cite{xxiivv}
\cite{plan-9}

\subsection[mega65]{MEGA65}

Macroification
\cite{understandable-computers}
\cite{fpga-based-mobile-phone}
\cite{megawat}

\subsubsection[megawat]{MegaWAT!?}

Was written in a semester as a university project\cite{megawat}.
Was demonstrated at linux.conf.au\cite{fpga-based-mobile-phone}.

\newpage

\section[references]{References}

\bibliography{bibliography}
\bibliographystyle{acm}

\end{document}
